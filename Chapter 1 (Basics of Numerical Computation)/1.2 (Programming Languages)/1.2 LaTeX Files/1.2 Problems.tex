\documentclass{article}

\usepackage{amsmath, amsfonts, amssymb, amsthm}
\usepackage[shortlabels]{enumitem}
\usepackage[margin=1in]{geometry}
\usepackage{setspace}
\usepackage{mathrsfs}
\usepackage{algpseudocode}
\renewcommand{\algorithmicrequire}{\textbf{Input:}}
\renewcommand{\algorithmicensure}{\textbf{Output:}}

% Custom Environments
\theoremstyle{definition}
\newtheorem{problem}{Problem}[subsection]

% Convience Functions
\newcommand{\courier}[1]{{\fontfamily{pcr}\selectfont#1}}

% Geometry and Formatting
\doublespacing
\renewcommand{\qedsymbol}{\(\blacksquare\)}
\pagenumbering{gobble}

\begin{document}
	\setcounter{section}{1}
	\setcounter{subsection}{2}
	
	% 1
	\setcounter{problem}{0}
	\begin{problem}
	Pick your favorite programming language. Is it interpreted? Is it compiled? What
	is (or are) the data type(s) for floating point numbers? Is there a built-in data
	type for vectors or matrices? Can the vectors (or matrices) be added? Can they
	be multiplied?
	\end{problem}
	
	% 2
	\newpage
	\setcounter{problem}{1}
	\begin{problem}
	What facilities are included in your favorite programming language for parallel
	programming? Are there libraries that provide these facilities? Is it possible
	to perform both shared memory and distributed memory computing in your
	language? With these libraries?
	\end{problem}
	
	% 3
	\newpage
	\setcounter{problem}{2}
	\begin{problem}
		Is your favorite programming language garbage collected? That is, is there automatic de-allocation of unusable objects? What are the advantages and disadvantages
		of garbage collection?
	\end{problem}
	
	% 4
	\newpage
	\setcounter{problem}{3}
	\begin{problem}
		The following function applies the function \(f\) to each item of a vector of items. Implement it in your favorite programming language.
		\begin{figure*}[h!]
			\begin{algorithmic}
				\Function{map}{f, x}
				\State \(y \gets \)\courier{ new array the same size as }\(x\)
				\For{\(i\) \courier{an index of }\(x\)}
					\State \(y_i\gets f(x_i)\)
				\EndFor
				\EndFunction
			\end{algorithmic}
		\end{figure*}
	
	
		Now implement it in your least favorite, or a previously unknown, programming
		language.
	\end{problem}
	
	% 5
	\newpage
	\setcounter{problem}{4}
	\begin{problem}
	Macros are pieces of code that \textit{transform other pieces of code} before compilation or execution. Does your favorite programming language have macros? What
	kinds of transformations can themacros in that language perform? If your favorite
	programming language does not have macros, or you discover another language
	that does have macros (such as Lisp or Julia), describe the macro facilities of
	that language.
	\end{problem}
	
	% 6
	\newpage
	\setcounter{problem}{5}
	\begin{problem}
	The C programming language was developed for implementing the Unix operating
	system. This led to many design decisions by the creators of the language,
	such as pointer arithmetic, and lack of array bounds. Explain why these decisions
	may have been necessary for their purpose at that time, and if you think those
	decisions help or hinder the creation of mathematical software now.
	\end{problem}
	
	% 7
	\newpage
	\setcounter{problem}{6}
	\begin{problem}
		Some languages are stack languages, such as Forth, Joy, and PostScript. Arguments
		for a “function” in a stack language do not have names, but the \(k-th\) input
		is the \(k-th\) item from the top of the stack. Download one of these languages and implement the “map” function of Problem 1.2.4 in that language.
	\end{problem}
	
\end{document}